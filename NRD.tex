\documentclass[10pt,letterpaper,twocolumn]{scrartcl}
\usepackage[margin=0.9in]{geometry}


\usepackage{lmodern}
\usepackage{comment}
\usepackage{amsmath}
\usepackage{amsfonts}
\usepackage{amssymb}
\usepackage{graphicx}
\usepackage{xcolor}

\usepackage[inner]{showlabels}


%opening
\title{Decoupling of a class of underactuated systems based on SMC Methogology}
\subtitle{ \textit{rev. 1 }}
\author{J.A. Ortega}

% operators

\DeclareMathOperator{\rank}{rank}

%definitions
\newcommand{\R}{\mathbb{R}}

\newcommand{\blue}[1]{\textcolor{blue}{#1}}
\newcommand{\red}[1]{\textcolor{red}{#1}}

% Document

\begin{document}

\maketitle

\section{Motivation}

Consider the system

\begin{equation} \tag{$\Sigma$} \label{eq:sys}
    \dot{x} = A x + B u,\quad x \in \R^n, u \in \R^m
\end{equation}

\noindent with one-degree of underactuation i.e., $ n = m + 1 $, and $A: \R^n \to \R^n, \quad B: \R^m \to \R^n$ are constants.\\

{Objective:} The goal is to design a sliding manifold $\sigma = C x$ to stabilize the system \eqref{eq:sys}.
As requirement $\sigma$ needs to be relative degree complete, i.e.,  $n$ with reference to the output  $u$.\\
The motivation of the objective is to solve the stabilization of a underactuated system in a simpler manner.
The methodology have two main steps: Design a $n$-RD sliding manifold $\sigma$ then, propose a $n$-order Continuous Sliding Mode Controller to stabilize it.
Note that, if the sliding-manifold has the same order that the system \eqref{eq:sys}, once $\sigma \equiv 0$ does not exist zero dynamics.

\section{Decoupling based on SMC}

First we define an output of the \eqref{eq:sys} as $s= C x$, s.t,
\[
    \dot{s} = C A x + C B u
.\]
\noindent in order to annihilate $CB$,

\[
    (CB)^\perp \dot{s} = (CB)^\perp C A x + \red{(CB)^\perp C B} u
.\]
\noindent which can be written as
\[
    \frac{d}{dt} \underbrace{\begin{bmatrix} (C B)^\perp & 0 \\ 0 & 1 \end{bmatrix}}_{J_1} \underbrace{\begin{bmatrix} s \\ \int s \, d \tau \end{bmatrix}}_{Y_1(t)} =  \underbrace{\begin{bmatrix} (CB)^\perp C A \\ C \end{bmatrix}}_{M_1} x(t)
\]

We continue computing the derivative of $\frac{d}{dt}J_1 Y_1(t) = M_1 x(t)$ in the same manner, until $\rank M_k = n$
\[
    \frac{d^2}{dt^2} J_1 Y_1(t) = M_1 A x + M_1 B u
,\]
\noindent then,
\[
    \frac{d^2}{dt^2} (M_1 B)^\perp J_1 Y_1(t) = (M_1 B)^\perp M_1 A x + \red{(M_1 B)^\perp M_1 B} u
.\]
\[
    \frac{d^2}{dt^2} \underbrace{\begin{bmatrix} (M_1 B)^\perp (C B)^\perp & 0 \\ 0 & (M_1 B)^\perp \end{bmatrix}}_{(M_1 B)^\perp J_1} \begin{bmatrix} s \\ \int s \, d \tau \end{bmatrix}= (M_1 B)^\perp M_1 A x
.\]

That can be rewritten as
\[
    \frac{d^2}{dt^2} \underbrace{\begin{bmatrix} (M_1 B )^\perp J_1 & 0 \\ 0 & 1  \end{bmatrix}}_{J_2}   \begin{bmatrix} s \\ \int s \, d \tau \\ \int \int s \, d \tau dy \end{bmatrix}= \underbrace{\begin{bmatrix}  (M_1 B)^\perp M_1 A \\ C \end{bmatrix}}_{M_2} x
.\]

Following the same procedure we will arrive to a system of $n$-order,

\begin{equation} \label{eq:compS}
    \frac{d^n}{dt^n} J_{n-1} \underbrace{\begin{bmatrix} s \\ \int s \, d \tau \\ \vdots \\ \int \dots \int s \, d \tau dy \end{bmatrix}}_{Y_{n-1}} = M_{n-1} A x + M_{n-1} B u
\end{equation}

Notice that by construction $J_k , \, \forall \, k = 1, \dots , n-1$ is diagonal and therefore also invertible, so, we can write \eqref{eq:compS} as,

\begin{equation}
    \frac{d}{dt}\begin{bmatrix} s^{(n-1)} \\ \vdots \\ \dot{s} \\ s \end{bmatrix} = J^{-1} M_{n-1} \left[ A x + B u \right]
\end{equation}

\section{Furuta Pendulum Stabilization with LTI model}

Consider the linearisation of the Furuta pendulum with the matrices
\[
    A = \begin{bmatrix} 0 & 0 & 1 & 0 \\ 0 & 0 & 0 & 1 \\ 0 & 80.3 & -45.8 & -0.93 \\ 0 & 122 & -44.1 & -1.4 \end{bmatrix}, \quad
    B = \begin{bmatrix} 0 \\ 0 \\ 83.4 \\ 80.3 \end{bmatrix}
.\]

\[
    \dot{s} = C A x + C B u 
.\]

\[
    \frac{d}{dt} \begin{bmatrix} s \\ \int s \end{bmatrix}  = \begin{bmatrix} CA \\ C \end{bmatrix}  x
.\]

\[
 \frac{d}{dt} Y_1(t) = M_1 x   
.\]

\[
    \frac{d}{dt} \begin{bmatrix} s \\ \int s \end{bmatrix}= \begin{bmatrix} 0 & 0 & 3 & 10 \\ 3 & 10 & 0 & 0 \end{bmatrix}  x   
.\]

\[
 \frac{d^2}{dt^2} Y_1(t) = M_1 A x + M_1 B u
.\]

\[
    \frac{d^2}{dt^2} (M_1 B)^\perp Y_1(t) = (M_1 B)^\perp M_1 A x 
.\]

\begin{equation}
    \frac{d^2}{dt^2} \begin{bmatrix} (M_1 B)^\perp & 0 & 0 \\ 0 & (M_1 B)^\perp & 0 \\ 0 & 0 & 1 \end{bmatrix}   \begin{bmatrix} s \\ \int s \\ \int \int s \end{bmatrix} = \begin{bmatrix} (M_1 B)^\perp M_1 A \\ C \end{bmatrix}   x 
\end{equation}

\[
    \ddot{s} = C A^2 x + C A B
.\]


\end{document}

